\documentclass{beamer}
\usepackage{fontspec}
\usepackage{xeCJK}
\usepackage{CJKutf8}
\usepackage{graphicx}
\usepackage{amsmath,amssymb}

\title{随机森林与模型的可解释性与预测能力权衡}
\author{Junchen Feng}
\date{2024-12-20}

\begin{document}

\maketitle

\begin{frame}{课程背景与目标}
  \begin{itemize}
    \item 通过随机森林为例,介绍集成学习(ensemble learning)
    \item 强调模型可解释性与预测能力之间的trade-off
  \end{itemize}
\end{frame}

\section{导入}
\begin{frame}{回顾与引入}
  \begin{itemize}[<+->]
    \item 回顾机器学习的核心目标:不仅在训练集上表现好,更需在新数据上有良好泛化能力
    \item 单一模型(如决策树)在复杂数据场景中局限性:易过拟合,泛化不足
  \end{itemize}
\end{frame}

\section{集成学习与随机森林原理}
\begin{frame}{集成学习的基本思想}
  \begin{itemize}[<+->]
    \item Ensemble Models:通过集成多个基学习器(Base Learners)提升整体性能
    \item 类比:专家委员会与单独专家的差异——多个决策对结果表决或平均
    \item 常见策略:Bagging与Boosting(本次重点在Bagging)
  \end{itemize}
\end{frame}

\section{Bagging与Boosting简述}

\begin{frame}{Bagging (Bootstrap Aggregating)}
  \begin{itemize}
    \item 基本思想:通过对训练集进行有放回抽样(Bootstrap)生成多个子数据集。
    \item 对每个子数据集训练一个基学习器(如决策树),最后对预测结果取平均(回归)或投票(分类)。
  \end{itemize}

  \textbf{数学表示:}

  假设有一个训练集 $D = \{(x_i, y_i)\}_{i=1}^N$。  
  Bagging通过抽样生成 $M$ 个数据子集 $D^{(m)}$,其中 $m = 1, 2, \ldots, M$。  
  对每个子集训练一个学习器 $h_m(x)$。  
  最终的Bagging预测为:
  \[
  H_{\text{bag}}(x) = \frac{1}{M} \sum_{m=1}^M h_m(x) \quad \text{(回归)}
  \]
  或对于分类问题:
  \[
  H_{\text{bag}}(x) = \text{majority\_vote}\big(h_1(x), h_2(x), \ldots, h_M(x)\big)
  \]
\end{frame}

\begin{frame}{Boosting}
  \begin{itemize}
    \item 基本思想:逐步训练一系列基学习器,每个新学习器都针对之前学习器的不足(错误样本)进行有偏重的再训练。
    \item 不同于Bagging的并行训练,Boosting是序列式构建学习器,并不断提升整体预测精度。
  \end{itemize}

  \textbf{数学表示(以AdaBoost为例):}

  给定训练集 $D = \{(x_i, y_i)\}_{i=1}^N$,初始样本权重为 $w_i^{(1)} = \frac{1}{N}$。  
  对于 $m = 1$ 到 $M$:
  \begin{enumerate}
    \item 基于当前权重分布训练基学习器 $h_m(x)$
    \item 计算加权错误率 $\epsilon_m = \sum_{i=1}^N w_i^{(m)} \mathbf{1}(h_m(x_i) \neq y_i)$
    \item 计算学习器权重 $\alpha_m = \frac{1}{2}\ln\frac{1-\epsilon_m}{\epsilon_m}$
    \item 更新权重分布:
    \[
    w_i^{(m+1)} = \frac{w_i^{(m)} \exp(-\alpha_m y_i h_m(x_i))}{Z_m}
    \]
    其中 $Z_m$ 是归一化因子
  \end{enumerate}

  最终Boosting预测为:
  \[
  H_{\text{boost}}(x) = \text{sign}\left(\sum_{m=1}^M \alpha_m h_m(x)\right)
  \]
\end{frame}

\begin{frame}{决策树的复习与不足}
  \begin{itemize}[<+->]
    \item 决策树优点:简单直观,可解释性强
    \item 单数要增加预测能力,就会太复杂,容易过拟合
    \item 能不能"聚"木成林
  \end{itemize}
\end{frame}

\begin{frame}{随机森林的构造原理(Random Forest)}
  \begin{itemize}
    \item Bagging步骤:对训练数据集有放回抽样,得到多个随机样本子集
    \item 每个子集训练一棵决策树,分裂时随机挑选特征子集,增加模型多样性
    \item 多棵树的预测结果采用投票(分类)或平均(回归)得到最终预测
  \end{itemize}
\end{frame}

\begin{frame}{随机森林的构造原理(Random Forest)}
  \begin{center}
    \textbf{随机性体现在哪里?}
  \end{center}
\end{frame}


\begin{frame}{随机森林的构造原理(Random Forest)}
  \begin{theorem}[随机森林的两重随机性]
    \begin{enumerate}
      \item 数据采样的随机性:Bootstrap抽样使每棵树看到不同的训练数据
      \item 特征选择的随机性:每次分裂时随机选择特征子集进行最优分裂
    \end{enumerate}
    这两重随机性保证了森林中每棵树都具有独特性,提高了整体模型的多样性和鲁棒性。
  \end{theorem}
\end{frame}


\begin{frame}{随机森林为何提高泛化性能?}
  \begin{itemize}
    \item 减少模型方差:个体决策树虽不稳定,但通过多数表决可"互相抵消"错误
    \item 不需对数据分布做严格假设,凭借随机性增加鲁棒性
  \end{itemize}
\end{frame}

\section{可解释性 vs. 预测能力的权衡}

\begin{frame}{可解释性与预测能力的平衡点}
  \begin{itemize}
    \item 决策树:可解释性强(清晰的规则路径),但单树预测能力有限
    \item 随机森林:性能提升(更高泛化能力),但单个预测路径较复杂,可解释性下降
    \item 实际问题中往往在解释性与预测性能之间寻找平衡
  \end{itemize}
\end{frame}

\begin{frame}{哪个特性更重要?}
  \begin{itemize}
    \item 金融信贷决策:需向客户和监管部门解释为什么拒绝贷款
    \item 高频金融交易:只要赚钱不需要可解释
  \end{itemize}
\end{frame}

\begin{frame}{Breiman的 "Two Cultures" }
  \begin{itemize}[<+->]
    \item 统计学文化:偏好可解释的参数模型,强调数据生成分布和明确的假设
    \item 机器学习文化:更注重预测性能,模型不一定有明确的分布假设
  \end{itemize}
\end{frame}

\begin{frame}{深层思考}
  \begin{itemize}[<+->]
    \item 面对复杂问题:究竟何为"好"模型?  
    \item 不同学科背景下,对解释与预测的侧重点不一样
    \item Breiman观点的启示:当我们谈论模型时,不仅谈精度,还要谈理念、假设与实用性
  \end{itemize}
\end{frame}



\end{document}